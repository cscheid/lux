% TODO: Decide on paper figures!!

%%%%%%%%%%%%%%%%%%%%%%%%%%%%%%%%%%%%%%%%%%%%%%%%%%%%%%%%%%%%%%%%%%%%%%%%%%%%%%

\documentclass[review,journal]{vgtc}
\usepackage{microtype}
\usepackage{clrscode}
\usepackage{mathptmx}
\usepackage{graphicx}
\usepackage{times}
\usepackage{xspace}
\onlineid{277}
\vgtccategory{Research}
\vgtcinsertpkg

\newcommand{\Shade}{\texttt{Shade}\xspace}
\title{Lux: Composable, High-Performance Visualization for the Web}
\author{Carlos Scheidegger}
\authorfooter{\item Carlos Scheidegger is with AT\&T Research, email: cscheid@research.att.com}
\shortauthortitle{Scheidegger: Lux}

\abstract{
We address the problem of writing succinct and expressive programs for high-performance, interactive visualizations on the web.
Modern web browsers provide WebGL, which exposes hardware-accelerated graphics and enables high-performance graphics on the web.
In this paper, we consider the WebGL API in the context of creating reusable building blocks for visualization programs, and identify some key shortcomings.
We present Lux, designed explicitly to address these shortcomings.
Lux is a \emph{domain-specific embedded language}, built around an optimizing source-to-source compiler.
Among other advantages, it gives users access to GPU programming without requiring them to write GLSL, the C-like vertex and fragment shading languages embedded in WebGL. 
Lux is capable of automatically \emph{deriving shaders}, which reduces needless repetition, increases source code reuse and allows a novel representation of scene hierarchies.
We provide experimental evidence of Lux's ability to provide higher-level abstractions than previously possible, with a fraction of the effort necessary with currently-available libraries.
}

\keywords{DSEL, WebGL, visualization toolkits}

\teaser{
\centering
\includegraphics[width=\linewidth]{fig/teaser/teaser.png}
\caption{LOOK AT MY AWESOME TEASER. JUST LOOK AT IT}
}

\begin{document}

\firstsection{Introduction}

\maketitle

Modern web browsers now expose much of the hardware-accelerated graphics API and computing power that only a few decades ago were exclusive of supercomputers.
Together with the ubiquity of graphics cards in the desktop and the rise of the world-wide web as a platform for application development and deployment, visualization researchers now have a clear opportunity to move high-performance visualizations to the world-wide web.
The current state-of-the-art API for three-dimensional web-based graphics is WebGL~\cite{webgl-spec}, an open standard based on OpenGL ES 2.0~\cite{opengles-spec} (here, ES stands for ``Embedded Systems'').

WebGL is a good API for exposing GPU capabilities, but it is far from an ideal language for \emph{programming}.
It is based on a state machine, and exposes much of its functionality via its vertex and fragment program languages. 
These languages allow for powerful, high-performance computations, but they have their own syntax and semantics separate from the host language (in this case, Javascript).
Therein lies a major problem: as we will show, WebGL libraries and applications written in pure Javascript suffer from ``impedance mismatch''.

In this paper, we contribute Lux, which seeks to remove one barrier for the wider adoption of WebGL as an effective programming library for high-performance graphics and visualization on the Web.
Lux was designed for WebGL, but the concepts generalize to other high-level graphics APIs such as OpenGL 4 and Direct3D 11, and other modern host languages such as C++, Java and Python. % FIXME this sentence belongs elsewhere

The observations leading to Lux and the design decisions behind the library are discussed in Section~\ref{sec:design}, after Section~\ref{sec:relatedwork} reviews the relevant literature and previous work and places it in context of Lux's features.
In Section~\ref{sec:evaluation}, we provide evidence that Lux scales well both in programming effort and in program performance.
We finish the paper with Section~\ref{sec:discussion}, a discussion of the implications of the presented work.

\section{Related Work\label{sec:relatedwork}}

\begin{figure*}
\includegraphics[width=\linewidth]{fig/pipeline_mismatch/pipeline_mismatch.jpg}
\caption{The fundamental impedance mismatch between the GPU pipeline and the visualization pipeline.\label{fig:mismatch}}
\end{figure*}

The world-wide web has recently emerged as the primary means of exchanging hypertext documents. 
As computing power has increased, traditional hypertext has become intertwined with rich media such as video and interactive graphics. 
Naturally, visualization and 3D graphics have recently received substantial attention.

% visualization for the Web
Heer and co-authors have done early work in the design of effective visualization and graphics infrastructure for the Web, including Java libraries such as Prefuse~\cite{Heer:2005:PAT} (and its followup Flare, based on Adobe's Flash runtime~\cite{Adobe:2011:Flash}).
Processing~\cite{Reas:2007:PAP} is a particularly successful example of a graphics and visualization library built on the Java platform.
Lux, in contrast, is based on WebGL, which does not require plugins and is an open standard~\cite{webgl-spec}.
Resig~\cite{Resig:2010:PJ} has recently released Processing.JS, a Javascript port of Processing.
Although very useful in practice, Processing.JS aims to recreate most of the Processing programming language and environment, and so has similar impedance mismatch problems when integrated with Javascript code.  In addition, Lux has better performance characteristics, as we argue in Section~\ref{sec:evaluation}.

More recently, Protovis~\cite{Bostock:2009:PAG} and $\textrm{D}^3$~\cite{Bostock:2011:DDD} have been proposed as modern embodiments of visualization libraries for the web. 
Both libraries follow the critical observation that visualization specifications should be \emph{declarative}: instead of specifying how to draw something, they should specify what is to be drawn. 
In other words, \emph{values} are emphasized over \emph{procedures}.
Lux takes a similar approach with its abstraction of the WebGL shading infrastructure.
Protovis (and portions of $\textrm{D}^3$) use Scalable Vector Graphics (SVG~\cite{svg-spec}) as their graphics target, and SVG is much closer to the rest of the ecosystem of the modern web.
In particular, it contains a full Document Object Model (DOM~\cite{dom-spec}). This has two main consequences.
The upside of having access to the DOM is that much of the programming model and software infrastructure in Javascript carries over to SVG representations: the infrastructure for graphics theming, interaction and so on are provided by the web browser itself. This is a big advantage.
The downside is that using SVG incurs a performance penalty that might be severe in large scale situations, as we describe in section~\ref{sec:evaluation}.
With respect to API design, having access to the DOM has another positive consequence: the distance to be reached between SVG programming and Javascript programming is smaller. For example, interactive visualization can be created by directly attaching event handlers to DOM elements (such as ``mouse click'', ``mouse move'', etc.)
As we argue in Section~\ref{sec:design}, with WebGL, this gap is wider.
% Visualization libraries for the Web
% Protovis, D3.

Iris Inventor~\cite{Strauss:1993:IIA} (which later became Open Inventor) was arguably among the original attempts to allow higher-level programming constructs in programs using OpenGL-likeAPIs.
Iris Inventor was built on top of IRIS GL, a proprietary ancestor of OpenGL from the late 1980s.
There are myriad similar libraries such as OpenSG, OpenSceneGraph, and NVIDIA's SceniX.
This happens in the WebGL ecosystem as well, with examples such as Three.JS, X3DOM, SceneJS~\cite{SceneJS}, glge, and SpiderGL~\cite{DiBenedetto:2010:SAJ}.
While all these libraries provide a large set of useful drawing primitives, composing appearance specifications remains challenging.
SceneJS generates some shader code automatically to fit the scene graph description, but custom shader generation in SceneJS is performed by textual source code injection.
This is not optimal for several reasons, including that managing variable naming conflicts is brittle, and some optimizations become impossible.

The observation that shading languages need their own layer of abstraction is not new. In fact, Cook's pioneering work in Shade Trees~\cite{Cook:1984:ST} comes largely from that observation.
McCool et al. later describe Sh, a shader metaprogramming library~\cite{McCool:2002:SM, McCool:2004:SA} which is an early example of embedding shading expressions in the host language (in their case, C++). 
%
Lux takes this model even further: most of its API is defined around denoting shading expressions as Javascript values. 
While Sh requires explicit assignment of expressions to vertex and fragment shaders, Lux, in constrast, defers these decisions to the compiler.
This increases reuse: for example, shader expressions can be hoisted to and from fragment and vertex programming stages as necessary.
Recently, Hanrahan and Foley have proposed Spark~\cite{Foley:2011:SMC}, a shader programming language which shares much of the modularity and composability design goals of Lux. 
In sharp contrast with Lux, Spark is its own language with custom syntax and semantics. 
Part of the appeal in Lux's design is that it bridges the gap between host and target languages.
One practical advantage of insisting in an embedded DSL design is the natural synergy that can be achieved. 
As we show in Section~\ref{sec:design}, this tighter integration can be leveraged by other parts of Lux, including a novel representation of hierarchical scene graphs.
In addition, DSELs can more easily be extended by host libraries: for example, Pellacini's automatic shader simplication technique~\cite{Pellacini:2005:UCA} could be implemented as a Javascript module for Lux. 

% Shading abstractions:
% Shade trees
% Shader meta-programming
% applications when shading abstractions are available:
% shader level of detail (pellacini)

\section{Design\label{sec:design}}

In this section we present the main aspects of Lux's design.
We start by describing what we call the \emph{fundamental impedance mismatch} between the computational building blocks of graphics APIs and the natural way to describe visualization specifications. 
Then, we show how Lux tackles this problem and describe the major aspects of the library.

\subsection{Visualization pipeline vs. GPU pipeline\label{sec:mismatch}}
Visualization specifications are, by and large, created by prescribing encodings of data elements into attributes of visual primitives. 
This observation dates all the way back to Bertin's original work~\cite{Bertin:1967:SOG}, and pervades all formal approaches to visualization design~\cite{Wickham:2009:GEG,Wilkinson:2005:TGO}.
These visual primitives can range from the very simplest, like a single dot, to not-as-simple, such as a boxplot, to arbitrarily complex, such as a polyline per data element in the case of parallel coordinates. 

In addition, the encoding of each attribute changes substantially from case to case. Consider the differences in the following examples.
In the case of a Grand Tour~\cite{Asimov:1985:TGT}, the visual primitive is a simple dot, but the position of the dot involves computing a projection of all coordinates of the multidimensional data point (we can think of each sample as a vector with $k$ coordinates) along the projection axes. 
In the case of parallel coordinates, the visual primitive itself is a more complicated polyline, while the position of the endpoints is simple to compute. 
Unlike the Grand Tour, there is a one-to-one correspondence between the position of an endpoint and a single coordinate of the multidimensional data point.
Finally, take the case of the traditional histogram. 
There, each visual primitive is a relatively simple rectangle with four vertices. 
Determining the height of the rectangle, however, involves a computation over \emph{all} input points. 
As Figure~\ref{fig:mismatch} illustrates, each visualization uses a different access pattern over the data.

This heterogeneity in access patterns is very much at odds with the natural way to specify graphics computations in graphics APIs such as WebGL, OpenGL and Direct3D. 
Referring again to Figure~\ref{fig:mismatch}, note that graphics APIs are modeled very closely to the underlying hardware, where data processing is done ``vertex-in, vertex-out''. 
If we specify vertex color, position, etc. in the most natural fashion, the attributes need to be passed to GPU in a one-to-one correspondence with the vertices. 
In other words, if a visual encoding requires an array of $n$ vertices, an array of exactly $n$ Cartesian positions will be necessary, together with an array of exactly $n$ colors, and so on. 
Now, compare the case of a Grand Tour visualization to that of parallel coordinates. 
A single frame of a Grand Tour rendering takes $n$ vertices, while a single frame of rendering parallel coordinates takes $n.k$ vertices, $n.k$ colors, etc. 
So if done in a straightforward manner, different visualizations require different preprocessing steps over the data, \emph{before} sending the data to the graphics card. 
This is unfortunate, and especially so in relatively slower languages like Javascript: we are stalling the fast graphics system by having the CPU touch every vertex at every frame.

Modern graphics cards, however, are \emph{programmable}. 
What this means is that the position and color of points and lines can be determined by the result of an actual program execution. 
This execution happens entirely in the GPU, in parallel (the position of a vertex is independent of the position of the next vertex). 
GPUs are quite parallel: a mobile device such as Apple's latest iPad has four graphics cores, and consumer video cards for desktops now have around two hundred ALUs executing GPU programs in parallel. 
To achieve fast graphics, then, the data should reside entirely on the graphics card, and the appropriate GPU programs for each visual primitive must be written. 
This is one fundamental problem which Lux seeks to address: how can we create a Javascript visualization library which must be based around producing GPU programs, while minimizing exposure to the quirks of the underlying GPU programming model?

\subsection{The Lux scene hierarchy: stacks of invertible, target-agnostic transformations}

When building a visualization, programmers invariably find themselves juggling many different spaces and coordinate systems in their heads.
The venerable OpenGL matrix stack is a great example of tackling this problem. 
In many visualization applications, however, programmers need a combination of linear and non-linear transformations (consider a cartographical application using a spherical Mercator projection that can be zoomed and panned).
A typical interaction task is to ask for the latitude and longitude under the mouse pointer. 
Mouse interactions happen on the \emph{host} language, but programs are executed on the GPU.
Without a single


\subsection{From GLSL to Lux}

\begin{figure}
\includegraphics[width=\linewidth]{fig/snippet_overview/overview.jpg}
\caption{Some snippets of Lux code, and their relationship to the
library structure.\label{fig:facetsnippet}}
\end{figure}

Before Lux is described in earnest, let us review an extremely simple program written in GLSL, WebGL's language for specifying GPU computations. 
The first thing to note is that when programming GLSL, there are in fact two programs one needs to write. 
The program which specifies vertex processing is known as the \emph{vertex shader}, and the program which specifies fragment processing is known as the \emph{fragment shader} (\emph{fragments} are pixel-sized pieces of triangles which are given final colors to be rendered on the screen). 
The vertex shader mostly computes the position of the vertices, and the fragment shader, the color of the triangles:

\begin{verbatim}
// Vertex shader
attribute vec4  world_pos, color_in;
uniform   mat4  transform;
varying   vec4  color;
void main()
{
    gl_Position = transform * world_pos;
    color = color_in;
}
// Fragment shader
varying vec4 color;
void main()
{
    gl_FragColor = color;
}
\end{verbatim}

It is important to notice the different types of declarations used for the different variables. 
``Attributes'' denote data which changes with every vertex, like position and color in the example. 
Attribute arrays are advanced in lockstep, so in a model with $n$ vertices, all attribute arrays must have exactly $n$ entries in memory. 
``Uniforms'' are values which are constant across the entire array of vertices, like the transformation from world to screen coordinates. 
``Varying'' variables denote values to be interpolated along vertices so they can be processed by the fragment programs. 
Note, in addition, that the fragment shader cannot access \texttt{color\_in} directly: it must go through a varying variable. 
This presents a significant reuse problem: if a user decides to change the data access patterns (which, as we have argued in Section~\ref{sec:mismatch}, is very likely to happen across different visualization techniques), shaders which specify roughly the same operation need to written differently. 
Worse still, the WebGL calls to change attribute arrays and uniforms are entirely different. 
Creating reusable shader libraries is then unlikely.

To address this issue, Lux defers the creation of GLSL shaders as long as possible. 
Instead, users create and manipulate \Shade values. \Shade values are Javascript objects which values in GLSL. 
For example, \texttt{Shade(5)} is a Javascript object that denotes the value \texttt{5};
\begin{verbatim}
function f(x, y) {
  return x.mul(y).add(5);
}
\end{verbatim}
\noindent is a Javascript function which takes two \Shade values $x$ and $y$, and returns a new \Shade value denoting $xy + 5$. (the method syntax is unfortunately inevitable, since Javascript does not support operator overloading). 
Lux provides a sizable library of builtin \Shade functions, which create and manipulate \Shade values (examples: \texttt{Shade.sin}, \texttt{Shade.cos}, \texttt{Shade.min}, \texttt{Shade.max}, \texttt{Shade.pow}, etc). 
The entire set of GLSL built-in functions~\cite{opengles-spec} is available for use as \Shade functions.

Users build visual primitives by combining \Shade expressions into larger building blocks. 
In its lowest-level layer, Lux has two main data structures: the \emph{model}, and the \emph{actor}. 
Consider the following complete (albeit trivial) Lux example:
\begin{verbatim}
gl = Lux.init();
model = Lux.model({
  type: "triangles",
  elements: 3,
  vertex: [[0,0], [1,0], [1,1]]
});
camera = Shade.Camera.perspective();
triangle = Lux.actor({ 
  model: model,
  appearance: {
    position: camera(model.vertex),
    color: Shade.color("white")
  }
});
Lux.Scene.add(triangle);
\end{verbatim}
The variable \texttt{model} holds a Lux \emph{model}, which says that the visual primitive will be made out of triangles (other possibilities include lines and points), and that there are exactly three vertices (so one total triangle). 
The variable \texttt{triangle} holds an \emph{actor}.
\texttt{Lux.actor} combines a model with \Shade values for the vertex position and color, calls the \Shade compiler to build the following GLSL program pair:
\begin{verbatim}
// vertex program
attribute vec2 _unique_name_2;
void glsl_name_6 (void) {
  ( gl_Position = 
    (mat4(1.60947, 0.0, 0.0, 0.0, 
          0.0, 2.41421, 0.0, 0.0, 
          0.0, 0.0, -1.00200, -1.0, 
          0.0, 0.0, -0.20020, 0.0) * 
    vec4(_unique_name_2, 0.0, 1.0)) ) ;
}
void main() { glsl_name_6(); }
// fragment program
void glsl_name_1 (void) {
  ( gl_FragColor = vec4(1.0, 1.0, 1.0, 1.0) ) ;
}
void main() {
  glsl_name_1() ;
}
\end{verbatim}
If we were to change the Lux program slightly:
\begin{verbatim}
gl = Lux.init(canvas);
model = Lux.model({
  type: "triangles",
  elements: 3,
  vertex: [[0,0], [1,0], [1,1]],
  color: [color("red"), color("green"), 
          color("blue")], // <-- NEW
});
camera = Shade.Camera.perspective();
triangle = Lux.actor({
  model: model, 
  appearance: {
    position: camera(model.vertex),
    color: model.color // <-- NEW
  }
});
Lux.Scene.add(triangle);
\end{verbatim}
Lux would then construct a shader with the appropriate new \emph{attribute} and \emph{varying} bindings:
\begin{verbatim}
// vertex program
attribute vec2 _unique_name_2;
attribute vec4 _unique_name_3; // <-- NEW
varying vec4 _unique_name_4; // <-- NEW
void glsl_name_6 (void) {
  ( gl_Position = 
    (mat4(1.60947, 0.0, 0.0, 0.0, 
          0.0, 2.41421, 0.0, 0.0, 
          0.0, 0.0, -1.00200, -1.0, 
          0.0, 0.0, -0.20020, 0.0) * 
    vec4(_unique_name_2, 0.0, 1.0)) ) ;
}
void glsl_name_8 (void) {
  ( _unique_name_4 = _unique_name_3; );
}
void main() { glsl_name_6(); glsl_name_8(); }
// fragment program
varying vec4 _unique_name_4; // <-- NEW
void glsl_name_1 (void) {
  ( gl_FragColor = _unique_name_4; );
}
void main() {
  glsl_name_1() ;
}
\end{verbatim}

Some users of Lux will never need to directly create actors from low-level models and appearances, since Lux provides higher-level \emph{marks} from which to build visualizations (the name ``mark'' comes from the Protovis concept of the same name~\cite{Bostock:2009:PAG}). 
Lux marks take \Shade value parameters which provide data-to-visual transformations, etc, and creates actors that can be directly added to the scene. 
A simple example of a scatterplot is shown in Figure~\ref{fig:facetsnippet}. 
We highlight that there is nothing special about marks as they are written in the Lux: they are simply convenience layers written on top of lower-level layers. 
This is an example of the \emph{composability} advantages of an DSEL approach: the visual specification can be composed seamlessly from independent parts.

%% Describe WebGL shortcomings.

%% Uniforms, attributes, varying; all have different APIs and slightly
%% incompatible uses, but they should all be fundamentally
%% interchangeable: they denote \emph{values of a certain type}.

\section{Case Studies and Evaluation\label{sec:evaluation}}

We start this section highlighting some of features of Lux through a
sequence of progressively more complex examples. We then compare the
library performance to current publicly available visualization
libraries. This section is not meant to be a comprehensive guide to
Lux's feature set, but rather a look at the consequences of its
design and impact on how visualization and graphics programs are written.

\subsection{Geographical data on the globe}

Start with a mesh; transform into a sphere; transform to mercator
coordinates; fetch texture; transform lat longs into spherical;

Map different attributes to different visuals. Dot size and color seem
like good ideas.

% Sphere: attribute or shader? Don't know, don't care
\begin{verbatim}
var sphere = function(long_segs, lat_segs)
{
    var uv_mesh = mesh(long_segs, lat_segs);
    var theta = mul(uv_mesh.u, 2 * PI);
    var phi = mul(uv_mesh.v, PI);
    var v = vec(mul(cos(theta), sin(phi)),
                cos(phi),
                mul(sin(theta), sin(phi)));
    return model({ vertex: v,
                   normal: v,
                   elements: uv_mesh.elements,
                   type: uv_mesh.type
                 });
}
\end{verbatim}

The procedure starts from a more basic Lux model (the \emph{mesh}),
which simply encodes a square patch in $[0,1]^2$ tesselated to the
desired detail. It then computes a sphere by transforming the patch
appropriately. \texttt{uv\_mesh.u} and \texttt{uv\_mesh.v} are not
arrays of coordinates, and neither are \texttt{cos(theta)},
\texttt{sin(theta)}, etc: they are shade tree expressions which encode
a value to be computed in a WebGL shader. This has important
consequences. For example, if any of the sphere fields end up being
used in expressions that must be computed by the fragment shader (for
example, a normal could be used in cube mapping), Lux will
automatically split the necessary computations between WebGL
\emph{attributes} and WebGL \emph{varying}s, the former only being
allowed in vertex shaders, and the latter only in fragment shaders. By
relieving the user from making this decision, Lux allows increased
code reuse and expressivity.

\subsection{Custom shapes and picking}

% Picking wedges: automatically creating new shaders from old.
Expressing WebGL shader computations directly in Lux objects has
another substantial advantage: Lux can manipulate these expressions
programatically, and automatically create derived expressions. A
particularly useful example comes from Lux's implementation of
picking functionality (that is, determining which object lies under
the cursor). Lux uses a popular approach to OpenGL picking, which is
to render the exact same scene to an offscreen buffer, only replacing
final fragment colors with \emph{object ids}. The pixel under the
cursor is then read back from the GPU into main memory.
%
Lux allows
users to write arbitrary shading expressions, which include
statements that \emph{discard} some fragments. Rendering complex shapes through appropriate use of the
discard statement is a popular way to achieve resolution
independence and pixel precision without tesselation, as for example
has been shown by Loop and Blinn for Bezier
curves~\cite{Loop:2005:RIC}.
%
To ensure that the picking buffer is drawn exactly in the
same way as the regular screen, it is necessary to write custom
paired shaders: one picking shader for every rendering
shader. Requiring these to be written manually is cumbersome,
inelegant, and error-prone. Lux, on the other hand,
\emph{automatically} calculates the picking shader from the rendering
shader. Any model with a \texttt{id} field in its appearance can be
drawn in both rendering and picking mode. 

\subsection{Performance}

\begin{figure*}
\includegraphics[width=\linewidth]{fig/performance/performance.jpg}
\caption{Comparing the performance of Lux to D3 and Protovis on two
standard benchmarks: a scatterplot matrix and a stacked graph}
\end{figure*}

Scale: Plot every zip7 centroid on top of an openstreetmap globe?
Rebuild zipscribble?

NEED TO SHOW PROCESSING.JS, PROTOVIS AND D3 AT THE VERY LEAST HERE.

WHAT ABOUT TRADITIONAL scivis?

volume rendering, streamlines? Those are both fairly challenging
because of the integrals and ODEs. Volume rendering is a little easier
because the end result is a single value per integral, while
streamlining requires access the full function after we integrate the
differential equation. How I wish I had geometry shaders...

Loops, geometry shaders.

SO, how do we reimplement geometry shaders in Lux?

\section{Discussion\label{sec:discussion}}

Animation, cost to recompile shaders, ease of programming.

Having a model of GLSL values in Javascript is important for
testing. Mention that this was used during the development of Lux
itself.


\section{Conclusion}

This paper argues that careful API design can have a large impact in
how visualization and 3D graphics software is written on the Web, and
contributes Lux as an example of how to address some of these
shortcomings.

We believe we have but scratched the surface of this area.
% So what do we do now with Lux?

\bibliographystyle{eg-alpha}
\bibliography{paper}

\end{document}
