\chapter{Quirks}

\section{\_shade\_type and WebGL objects}

Some \lux objects are WebGL objects with extra slots associated with
them. Originally, this let me use them directly in WebGL calls while
storing additional information about them. It is no longer clear this
is a good idea, but they're here for now.

These objects store a \texttt{\_shade\_type} slot which currently can
take the values ``attribute\_buffer'', ``element\_buffer'', ``render\_buffer'' and
``texture''. Right now, the main use for these is to be able to create
a texture with Lux.texture and use it directly in a Shade
expression, without having to go through creating a sampler
object. This is a good thing.

It is probably a better idea, nevertheless, to have a prototype object
``WebGLObject'' which contains all the necessary calls so I can
dispatch on it.

\section{RTTI}

It is very convenient to use runtime type checking to get
polymorphism, but it seems like it tends to proliferate along the
code. I should try to consolidate all these calls in a single API of
some sort.
